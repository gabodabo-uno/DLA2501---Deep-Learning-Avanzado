\documentclass[12pt,a4paper]{article}
\usepackage[utf8]{inputenc}
\usepackage[spanish]{babel}
\usepackage{graphicx}
\usepackage[left=2cm,right=2cm,top=2cm,bottom=2cm]{geometry}
\usepackage{authblk} % Para el formato de los autores
%\usepackage{apacitet} %Creo que puedo omitir esta linea
\usepackage{natbib}
%\usepackage[natbibapa]{apacite}
\usepackage{mathtools} %Para poder poner dos ecuaciones lado a lado
\usepackage{float}
%\usepackage{dirtytalk}
%\usepackage{amsmath} %Para poder dividir en dos lineas las ecuaciones
\usepackage{threeparttable}
\usepackage{xcolor} %Para colores en el texto
%\DeclareMathOperator*{\E}{\mathbb{E}} %Tengo que tener el paquete amsmath
\usepackage{booktabs} % Para que pueda usar los comandos que salen con el output de Python 
\usepackage{longtable} %Para usar longtable si es que en  Python se


\usepackage{neuralnetwork}

%###############
%Estos dos van juntos para crear un símbolo
\usepackage{amsmath,amssymb}
\DeclareMathOperator{\E}{\mathbb{E}}
%###############

\title{Midterm Exam: Advanced Deep Learning}
\author[1]{Gabriel Santiago Belevan}
%\affil[1]{Facultad de Economía UPC \\email: u201713734@upc.edu.pe}

\date{\today}


\begin{document}
\maketitle

    \begin{neuralnetwork}[height=4]
        \newcommand{\x}[2]{$x_#2$}
        \newcommand{\y}[2]{$\hat{y}_#2$}
        \newcommand{\hfirst}[2]{\small $h^{(1)}_#2$}
        \newcommand{\hsecond}[2]{\small $h^{(2)}_#2$}
        \inputlayer[count=3, bias=true, title=Input\\layer, text=\x]
        \hiddenlayer[count=4, bias=false, title=Hidden\\layer 1, text=\hfirst] \linklayers
        \hiddenlayer[count=3, bias=false, title=Hidden\\layer 2, text=\hsecond] \linklayers
        \outputlayer[count=2, title=Output\\layer, text=\y] \linklayers
    \end{neuralnetwork}





\end{document}