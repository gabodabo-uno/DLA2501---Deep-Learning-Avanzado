\documentclass[10pt]{beamer}
\usetheme{CambridgeUS}
\usepackage[utf8]{inputenc}
%\usepackage[spanish]{babel}
\usepackage{amsmath}
\usepackage{threeparttable} 
\usepackage{amsfonts}
\usepackage{amssymb}
\usepackage{natbib}
\usepackage{mathtools}

\DeclareMathOperator*{\argmax}{arg\,max}

\usepackage{listings} %Para poner Py code 
% https://es.overleaf.com/learn/latex/Code_listing

\usepackage{xcolor} 
\usepackage{listings} 
\usepackage{setspace} 

\definecolor{backcolour}{rgb}{0.95,0.95,0.92}
\definecolor{codegray}{rgb}{0.5,0.5,0.5}

\definecolor{NormalText}{HTML}{FFFFFF} 
\definecolor{Coment}{HTML}{999999} 
\definecolor{String}{HTML}{0DC042} 
\definecolor{Number}{HTML}{FAED5C} 
\definecolor{Keyword}{HTML}{C670E0} 
\definecolor{Builtin}{HTML}{FAB16C} 
\definecolor{Definition}{HTML}{57D6E4} 
\definecolor{Instance}{HTML}{EE6772} 
 
\lstdefinestyle{mystyle}{
    backgroundcolor=\color{backcolour},   
    commentstyle=\color{Coment},
    keywordstyle=\color{Keyword},
    numberstyle=\tiny\color{codegray},
    stringstyle=\color{String},
    basicstyle=\ttfamily\footnotesize,
    breakatwhitespace=false,         
    breaklines=true,                 
    captionpos=b,                    
    keepspaces=true,                 
    numbers=left,                    
    numbersep=5pt,                  
    showspaces=false,                
    showstringspaces=false,
    showtabs=false,                  
    tabsize=2
}

\lstset{style=mystyle}


%###############
%Estos dos van juntos para crear un símbolo
\usepackage{amsmath,amssymb}
\DeclareMathOperator{\E}{\mathbb{E}}
\DeclareMathOperator*{\argmin}{arg\,min}
%###############

\usepackage{hyperref} %Para correos y urls
\hypersetup{
    colorlinks=true,
    linkcolor=blue,
    filecolor=magenta,      
    urlcolor=cyan,
    pdftitle={Overleaf Example},
    pdfpagemode=FullScreen,
    }

\title[Midterm Exam: DLA2501]{Midterm Exam: Advanced Deep Learning}
\author{Gabriel Santiago Belevan}
%\institute[RIEF]{The Research Institute in Economics and Finance}
%\titlegraphic{
%\includegraphics[width=2cm]{Logo.jpg}
%}

\begin{document}

\begin{frame}
\titlepage
\end{frame}

\begin{frame}{Problematic}
Perú es una economía emergente con un crecimiento sostenido en los últimos años. Sin embargo, no ha sido ajeno a los efectos del cambio climático:\\
\vspace{0.5cm}
\textbf{Efectos del Fenómeno de El Niño en la economía peruana.}
\begin{itemize}
    \item PBI $\downarrow 1.5\%$ estimado  (\href{https://acortar.link/KPBUJg}{EY, 2023})
    \item +245K familias afectadas (\href{https://acortar.link/TkdufF}{ONU, 2023})
\end{itemize}

\vspace{0.5cm}
\textbf{Propuesta de valor:}
Construcción de modelo de predicción de huaicos en el Perú.

\end{frame}

\begin{frame}{Data}
En cuanto a información por utilizar, seguimos al Ministerio de Agro y Riego del Perú (\href{https://acortar.link/PQVWFQ}{MIDAGRI})

La información proveendrá de la base de datos de World Bank Database: \href{https://climateknowledgeportal.worldbank.org/country/peru/climate-data-historical}{climateknowledgeportal.worldbank}. Gridded time-series dataset que contiene: 
\begin{itemize}
    \item Temperatura del aire y del oceano
    \item Precipitaciones
    \item Viento
\end{itemize}
\vspace{0.5cm}

\textbf{Retos de la }\textcolor{red}{\textbf{data}} \textbf{y} \textcolor{blue}{\textbf{metodología}}: 
\begin{itemize}
    \item \textcolor{red}{Formato de información desconocido}
    \item\textcolor{red}{¿Estmos seguros que la información es legible con software libre?}
    \item \textcolor{blue}{¿Cómo juntamos diferntes tipos de datasets como series de tiempo, imagenes, features ad-hoc de regiones (e.g., contaminación, número de ciudades, tipo de terreno, etc.)? Si fuese el caso de tratar de predecir cancer: usar imagenes e historial clínico.}
\end{itemize}

\end{frame}

\begin{frame}{Hypothesis}
    Nuestra hipotesis es que el fenómeno del Niño y las regiones afectadas por los huaicos son procesos modelables por medio de redes neuronlaes convolucionales y LSTM.
    \begin{itemize}
        \item Convolucionales: Las precipitaciones en el cuadrante $i$ afectan no solo a dicho cuadrante.
        \item LSTM: Las precipitaciones tienen una componente tendencial de serie de tiempo que puede ser modelada con este tipo de capas.
    \end{itemize}
\end{frame}


\end{document}
